\documentclass[]{book}
\usepackage{lmodern}
\usepackage{amssymb,amsmath}
\usepackage{ifxetex,ifluatex}
\usepackage{fixltx2e} % provides \textsubscript
\ifnum 0\ifxetex 1\fi\ifluatex 1\fi=0 % if pdftex
  \usepackage[T1]{fontenc}
  \usepackage[utf8]{inputenc}
\else % if luatex or xelatex
  \ifxetex
    \usepackage{mathspec}
  \else
    \usepackage{fontspec}
  \fi
  \defaultfontfeatures{Ligatures=TeX,Scale=MatchLowercase}
\fi
% use upquote if available, for straight quotes in verbatim environments
\IfFileExists{upquote.sty}{\usepackage{upquote}}{}
% use microtype if available
\IfFileExists{microtype.sty}{%
\usepackage{microtype}
\UseMicrotypeSet[protrusion]{basicmath} % disable protrusion for tt fonts
}{}
\usepackage[margin=1in]{geometry}
\usepackage{hyperref}
\hypersetup{unicode=true,
            pdftitle={GAMA Model Documentations},
            pdfauthor={SS},
            pdfborder={0 0 0},
            breaklinks=true}
\urlstyle{same}  % don't use monospace font for urls
\usepackage{natbib}
\bibliographystyle{plainnat}
\usepackage{color}
\usepackage{fancyvrb}
\newcommand{\VerbBar}{|}
\newcommand{\VERB}{\Verb[commandchars=\\\{\}]}
\DefineVerbatimEnvironment{Highlighting}{Verbatim}{commandchars=\\\{\}}
% Add ',fontsize=\small' for more characters per line
\usepackage{framed}
\definecolor{shadecolor}{RGB}{248,248,248}
\newenvironment{Shaded}{\begin{snugshade}}{\end{snugshade}}
\newcommand{\AlertTok}[1]{\textcolor[rgb]{0.94,0.16,0.16}{#1}}
\newcommand{\AnnotationTok}[1]{\textcolor[rgb]{0.56,0.35,0.01}{\textbf{\textit{#1}}}}
\newcommand{\AttributeTok}[1]{\textcolor[rgb]{0.77,0.63,0.00}{#1}}
\newcommand{\BaseNTok}[1]{\textcolor[rgb]{0.00,0.00,0.81}{#1}}
\newcommand{\BuiltInTok}[1]{#1}
\newcommand{\CharTok}[1]{\textcolor[rgb]{0.31,0.60,0.02}{#1}}
\newcommand{\CommentTok}[1]{\textcolor[rgb]{0.56,0.35,0.01}{\textit{#1}}}
\newcommand{\CommentVarTok}[1]{\textcolor[rgb]{0.56,0.35,0.01}{\textbf{\textit{#1}}}}
\newcommand{\ConstantTok}[1]{\textcolor[rgb]{0.00,0.00,0.00}{#1}}
\newcommand{\ControlFlowTok}[1]{\textcolor[rgb]{0.13,0.29,0.53}{\textbf{#1}}}
\newcommand{\DataTypeTok}[1]{\textcolor[rgb]{0.13,0.29,0.53}{#1}}
\newcommand{\DecValTok}[1]{\textcolor[rgb]{0.00,0.00,0.81}{#1}}
\newcommand{\DocumentationTok}[1]{\textcolor[rgb]{0.56,0.35,0.01}{\textbf{\textit{#1}}}}
\newcommand{\ErrorTok}[1]{\textcolor[rgb]{0.64,0.00,0.00}{\textbf{#1}}}
\newcommand{\ExtensionTok}[1]{#1}
\newcommand{\FloatTok}[1]{\textcolor[rgb]{0.00,0.00,0.81}{#1}}
\newcommand{\FunctionTok}[1]{\textcolor[rgb]{0.00,0.00,0.00}{#1}}
\newcommand{\ImportTok}[1]{#1}
\newcommand{\InformationTok}[1]{\textcolor[rgb]{0.56,0.35,0.01}{\textbf{\textit{#1}}}}
\newcommand{\KeywordTok}[1]{\textcolor[rgb]{0.13,0.29,0.53}{\textbf{#1}}}
\newcommand{\NormalTok}[1]{#1}
\newcommand{\OperatorTok}[1]{\textcolor[rgb]{0.81,0.36,0.00}{\textbf{#1}}}
\newcommand{\OtherTok}[1]{\textcolor[rgb]{0.56,0.35,0.01}{#1}}
\newcommand{\PreprocessorTok}[1]{\textcolor[rgb]{0.56,0.35,0.01}{\textit{#1}}}
\newcommand{\RegionMarkerTok}[1]{#1}
\newcommand{\SpecialCharTok}[1]{\textcolor[rgb]{0.00,0.00,0.00}{#1}}
\newcommand{\SpecialStringTok}[1]{\textcolor[rgb]{0.31,0.60,0.02}{#1}}
\newcommand{\StringTok}[1]{\textcolor[rgb]{0.31,0.60,0.02}{#1}}
\newcommand{\VariableTok}[1]{\textcolor[rgb]{0.00,0.00,0.00}{#1}}
\newcommand{\VerbatimStringTok}[1]{\textcolor[rgb]{0.31,0.60,0.02}{#1}}
\newcommand{\WarningTok}[1]{\textcolor[rgb]{0.56,0.35,0.01}{\textbf{\textit{#1}}}}
\usepackage{longtable,booktabs}
\usepackage{graphicx,grffile}
\makeatletter
\def\maxwidth{\ifdim\Gin@nat@width>\linewidth\linewidth\else\Gin@nat@width\fi}
\def\maxheight{\ifdim\Gin@nat@height>\textheight\textheight\else\Gin@nat@height\fi}
\makeatother
% Scale images if necessary, so that they will not overflow the page
% margins by default, and it is still possible to overwrite the defaults
% using explicit options in \includegraphics[width, height, ...]{}
\setkeys{Gin}{width=\maxwidth,height=\maxheight,keepaspectratio}
\IfFileExists{parskip.sty}{%
\usepackage{parskip}
}{% else
\setlength{\parindent}{0pt}
\setlength{\parskip}{6pt plus 2pt minus 1pt}
}
\setlength{\emergencystretch}{3em}  % prevent overfull lines
\providecommand{\tightlist}{%
  \setlength{\itemsep}{0pt}\setlength{\parskip}{0pt}}
\setcounter{secnumdepth}{5}
% Redefines (sub)paragraphs to behave more like sections
\ifx\paragraph\undefined\else
\let\oldparagraph\paragraph
\renewcommand{\paragraph}[1]{\oldparagraph{#1}\mbox{}}
\fi
\ifx\subparagraph\undefined\else
\let\oldsubparagraph\subparagraph
\renewcommand{\subparagraph}[1]{\oldsubparagraph{#1}\mbox{}}
\fi

%%% Use protect on footnotes to avoid problems with footnotes in titles
\let\rmarkdownfootnote\footnote%
\def\footnote{\protect\rmarkdownfootnote}

%%% Change title format to be more compact
\usepackage{titling}

% Create subtitle command for use in maketitle
\newcommand{\subtitle}[1]{
  \posttitle{
    \begin{center}\large#1\end{center}
    }
}

\setlength{\droptitle}{-2em}
  \title{GAMA Model Documentations}
  \pretitle{\vspace{\droptitle}\centering\huge}
  \posttitle{\par}
  \author{SS}
  \preauthor{\centering\large\emph}
  \postauthor{\par}
  \predate{\centering\large\emph}
  \postdate{\par}
  \date{2018-03-27}


\begin{document}
\maketitle

{
\setcounter{tocdepth}{1}
\tableofcontents
}
\hypertarget{species}{%
\chapter{Species}\label{species}}

 This is class css

Dutch nationals can today vote in what is likely to be the last
referendum for some time -- whether or not to give far-reaching powers
to the two security services to gather information, particularly via
phone and internet taps. The law is due to come into effect in May and
has already been passed by both houses of parliament. Nevertheless, over
400,000 people signed a petition calling for a referendum, hence today's
vote. The referendum is advisory and comes a month after the government
agreed to abolish the principle of advisory referendums altogether.
However, home affairs minister Kajsa Ollongren has already promised to
`take the result seriously'. An opinion poll published by current
affairs show EenVandaag on Tuesday suggests that a majority of voters
are likely to vote in favour of the legislation. Some 53\% of people who
said they are likely to vote, now plan to vote yes. But the result is
divided sharply along age lines, with 60\% of people over the age of 55
planning to vote in favour, compared with 41\% of the under-35s.

\hypertarget{imports}{%
\chapter{Imports}\label{imports}}

\begin{Shaded}
\begin{Highlighting}[]
\BuiltInTok{int} \OperatorTok{=} \DecValTok{10}\OperatorTok{;}
\BuiltInTok{list}\OperatorTok{<}\BuiltInTok{int}\OperatorTok{>}\NormalTok{ l }\OperatorTok{<-}\NormalTok{ [ }\DecValTok{1}\NormalTok{,}\DecValTok{2}\NormalTok{,}\DecValTok{3}\NormalTok{,}\DecValTok{4}\NormalTok{]}\OperatorTok{;}

\NormalTok{reflex aname when:(x}\OperatorTok{=}\DecValTok{10}\NormalTok{)\{}

\NormalTok{\}}
\end{Highlighting}
\end{Shaded}

\hypertarget{functions}{%
\chapter{Functions}\label{functions}}

Most of the Dutch papers have now published lengthy articles examining
the pros and cons of the new legislation. Wednesday's referendum was
initiated by five students from Amsterdam, whose four main issues with
the new law are examined by Trouw. The students say the law will allow
the state to listen in on entire neighbourhoods, hence the name `dragnet
law'. But, says Trouw, the law only allows the `tapping, taping and
listening in' of people who are a danger to national security and the
home affairs minister will have to approve the taps. Data belonging to
anyone caught up inadvertently must be removed as quickly as possible,
Trouw says, adding that the home affairs minister considers the law
contains enough guarantees to prevent indiscriminate tapping.

\hypertarget{reflexes}{%
\chapter{Reflexes}\label{reflexes}}

The students are also worried about the power of the intelligence
services to hack any device people may have in their homes, including
smart fridges, watches or cars. This will be allowed under the new law,
Trouw says, adding the secret services can even hack the devices via
another person's computer. But again, the services need permission from
the minister. Another worry the students have is that the legislation
will allow the secret services to set up their own DNA bank and their
powers to compare any DNA found at `locations of interest' with samples
in their own DNA bank. DNA profiles will be kept for five years, which
can be stretched to 30 with the permission of the minister. Ministers do
concede that DNA from people who are no threat to national security
might end up in this database and have said `a note must be made of
this', Trouw said. The final problem the students have with the law is
that raw data, gleaned from a wide variety of data bases can be shared
with foreign intelligence services. Again Trouw points out that sharing
data is subject to a number of checks and balances, such as the
protection of human rights and the protection of the data itself. Again,
the minister must give permission. Nevertheless, critics point out, this
does mean that data could end up in unfriendly foreign hands via third
parties,

\hypertarget{experiment}{%
\chapter{Experiment}\label{experiment}}

The Parool also looked at these issues but can only reiterate that while
the new law can pose a threat to people's privacy there are, on paper, a
number of guarantees that will prevent this from happening. The
Volkskrant interviewed cyber expert Huib Modderkolk. According to
Modderkolk the increased powers of the intelligence services are not
going to lead to a police state. `The most serious breach of privacy
citizens have to contend is the continual scanning of car registration
plates,' he said. `Google keeps a track of your whereabouts, Albert
Heijn knows which products we buy and the tax office links all sorts of
databases,' he told the paper. Modderkolk does not believe the
large-scale tapping will be a problem but thinks hacking of specific
targets will become more frequent.

\hypertarget{display}{%
\chapter{Display}\label{display}}

The cyber expert also doubts that `great haystacks of information' will
help defeat terrorism. `French intelligence services have the most
far-reaching powers in Europe and yet most attacks happened on French
soil,' he said. `Digital information on its own will not prevent
attacks. But if terrorists are using coded apps\ldots{} it would limit
the work of the intelligence services if they are not allowed to look at
that,' the paper quotes him as saying.

Read more at DutchNews.nl: As Dutch vote in referendum, poll shows older
people back new internet tapping laws
\url{http://www.dutchnews.nl/news/archives/2018/03/as-dutch-vote-in-referendum-poll-shows-older-people-back-new-internet-tapping-laws/}


\end{document}
